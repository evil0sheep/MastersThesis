\chapter{Conclusion}

The hardware needed to provide high quality 3D user interfaces is finally becoming available to consumers, and  the diversity of this hardware is growing rapidly. Currently, applications must integrate with each piece of hardware individually, and there is no widely adopted mechanism to allow multiple applications to share this hardware in a meaningful way. This limits support for hardware and creates a fragmented software ecosystem facing serious barriers to maturing properly.

This thesis proposes that these problems facing 3D user interfaces can be solved in the same way that they were solved for 2D user interfaces: by providing a system level 3D interface abstraction through the windowing system. It presents a viable architecture for 3D windowing systems which integrates well with existing windowing infrastructure, and demonstrates this with an implementation of this architecture, called Motorcar, built on top of an existing windowing system, Wayland. 

The systems and concepts presented here are intended to form a basis for further research into the field and to provide a functioning open source implementation which other components can be developed around and integrated with. This represents but one of many steps in the long process of bringing functional, modular, general purpose 3D user interfaces to every day computer users,  but it is also an important one. Hopefully, with further work, our interactions with computers will one day be freed from their two dimensional constraints and brought into the three dimensional space in which we interact with everything else.

